\documentclass{article}

\title{Resumo do Projeto do Foruns}
\author{
  André Plancha, 105289 \\
  \email{Andre\_Plancha@iscte-iul.pt}\\
  Allan Kardec Rodrigues, 103380 \\
  \email{aksrs@iscte-iul.pt} \\
  Rui Chaves, 104914 \\
  \email{rfpcs1@iscte.pt}\\
  \vspace{30pt}
}
\date{\today}

\begin{document}
  \thispagestyle{empty}
  \maketitle
  \pagebreak
  \section{Introdução}
  \textbf{Karcheve ou Charkarl} Estamos planejando desenvolver uma plataforma de fóruns altamente interativa e personalizável com o auxílio do Django. Esta aplicação proporcionará um ambiente robusto e amigável para a troca de ideias e informações entre os membros da comunidade. Com recursos avançados de moderação, categorização intuitiva e uma interface intuitiva, nosso objetivo é criar uma experiência envolvente para os usuários, promovendo discussões construtivas e interação significativa.
  \pagebreak
  \section{Funcionalidades do Fórum}
  
  \subsection{Threads}
  Os usuários terão a capacidade de criar e participar de discussões em threads dedicadas.
  
  \subsection{Follow Ups}
  Os usuários poderão seguir threads de interesse e receber notificações sobre atualizações.
  
  \subsection{Usuários}
  O fórum contará com diferentes níveis de permissões para usuários, incluindo admins, mods, usuários verificados e normais.
  
  \subsection{Autenticação}
  O fórum terá um sistema de autenticação que inclui login, registro e logout de contas de usuário. Também será possível acessar algumas áreas sem a necessidade de login.
  
  \subsection{Página Principal}
  Uma página inicial será disponibilizada para navegação rápida e acesso facilitado às principais seções do fórum.
  
  \subsection{Pesquisa}
  Uma funcionalidade avançada de pesquisa estará presente, permitindo a busca por texto completo e suporte a expressões regulares.
  
  \subsection{Perfil do Usuário}
  Cada usuário terá uma página de perfil personalizada com informações sobre os threads que criou e participou.
  
  \subsection{Stalk}
  Os usuários poderão visualizar as postagens de outros usuários em um formato de acompanhamento.
  
  \subsection{Relatórios e Remoção de Usuários}
  Os usuários terão a capacidade de relatar comportamentos inadequados e os administradores poderão remover contas problemáticas.
  
  \subsection{Comentários dos Usuários}
  Os usuários poderão comentar sobre as postagens de outros usuários, facilitando a interação direta.
  
  \subsection{Citações}
  Será possível citar outras postagens, incluindo data de publicação e autor.
  
  \subsection{Remoção e Edição}
  Os usuários terão a capacidade de remover ou editar postagens, seguindo regras e prazos predefinidos.
  
  \subsection{Assinatura Personalizada}
  Os usuários poderão adicionar uma assinatura personalizada às suas postagens.
  
  \subsection{Foto de Perfil e Biografia}
  Os usuários poderão fazer upload de fotos de perfil e compartilhar informações pessoais em suas biografias.
  
  \subsection{Título do Thread}
  Cada thread poderá ter um título descritivo.
  
  \subsection{Formatação de Texto}
  Haverá suporte a uma formatação de texto estilo Markdown para criar postagens mais ricas.
  
  \subsection{Classificação de Threads}
  Os usuários poderão classificar threads por critérios como mais recentes, mais antigos, mais citados e por usuário.
  
  \subsection{Filtragem}
  Haverá a opção de filtrar threads com base em critérios como usuários ou palavras-chave no título ou conteúdo.
  
  \subsection{Seguidores}
  Os usuários poderão seguir threads para receber notificações sobre atualizações.
  
  \subsection{RSS}
  O fórum contará com suporte a feeds RSS.
  
  \subsection{Paginação}
  As listas de threads serão paginadas para facilitar a navegação.
  
  \section{Conclusão}
  Essas funcionalidades foram cuidadosamente planejadas para oferecer uma experiência completa aos usuários do fórum.
  
  

\end{document}
